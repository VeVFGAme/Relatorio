\chapter{Resultados}
    
    \begin{itemize}
        \item Documentação
        
            Com base na documentação presente no repositório do \textit{Ignitest}\footnote{https://github.com/VerVal-2016-1/Framework}, responda:
            \begin{itemize}
                \item \textbf{Com as informações presentes na seção "Instalação", foi possível instalar o \textit{Ignitest} sem dificuldades?}
                    \subitem \textit{Sim.}

                \item \textbf{Com as informações presentes na seção "Instruções", foi possível entender o funcionamento do \textit{Ignitest}?}
                    \subitem \textit{Sim.}
            \end{itemize}

        \item Resultados
            \begin{itemize}
                \item Tempo
                    \begin{enumerate}
                        \item \textbf{Qual o tempo gasto para testar uma determinada classe, usando o método padrão?}

                            \subitem \textit{Para escrita de 25 casos de teste foram gastos 56 minutos.}              
              
                        \item \textbf{Qual o tempo gasto para testar a mesma classe com o auxílio do \textit{Ignitest}?}
              
                            \subitem \textit{Para a escrita dos mesmos casos de teste foram gastos 23 minutos.}
                        
                    \end{enumerate}
                \item Dificuldades
                    \begin{enumerate}
                        \item \textbf{Existem dificuldades em testar usando o método padrão? Se sim, quais?}
                            \subitem \textit{Não se pode dizer que são dificuldades, mas há necessidade constante de fazer importação da classe o que faz com que leve mais tempo o processo de teste, e o testador tem que ser disciplinado para não haja duplicação de código.}

                        \item \textbf{Essas dificuldades foram reduzidas, ou sanadas, com o uso do \textit{Ignitest}?}
                            \subitem \textit{Com o uso do Ignitest melhorou principalmente a diminuição de código duplicado e da preocupação com requires.} 
                        
              
                        \item \textbf{Novas dificuldades apareceram com o uso do \textit{Ignitest}? Se sim, quais?}
                        
                            \subitem \textit{Não.}
        
                    \end{enumerate}
                \item Quantidade de código
                    \begin{enumerate}
                        \item \textbf{Usando o método padrão, quantas linhas de código foram escritas para o teste da classe escolhida?}
              
                            \subitem \textit{Foi escrito código para 25 casos de testes com um total de 537 linhas}

              
                        \item \textbf{Com o uso do \textit{Ignitest}, quantas linhas de código foram escritas para o teste da mesma classe?}
              
                            \subitem \textit{Foi escrito código para 25 casos de testes com um total de 239 linhas}
              
                    \end{enumerate}

                \item Erros
                    \begin{enumerate}
                        \item \textbf{Usando o método padrão, quantos erros ao inserir comandos você cometeu?}
                        
                        \subitem \textit{Nenhum.}
                        
              
                        \item \textbf{Com o \textit{Ignitest}, quantos erros ao inserir comandos você cometeu?}
              
                        \subitem \textit{Nenhum.}
                        
                    \end{enumerate}
            \end{itemize}
        
        \item \textit{Feedback}
            \begin{itemize}
                \item \textbf{Como foi sua experiência usando o \textit{Ignitest}?}
            
           \subitem \textit{Foi boa. Foi mais fácil de realizar testes e teve menos duplicação de código}.
          
                \item \textbf{Durante o uso do \textit{Ignitest}, recebeu alguma mensagem de erro?}
                
                    \subitem \textit{Não.}
           
                \item \textbf{Durante o uso, o \textit{Ignitest} deixou de funcionar em algum momento?}
                
                    \subitem \textit{Não.}
           
                \item \textbf{Possui alguma sugestão de melhoria?}
                
                    \subitem \textit{Um comando para criar todas as classes de teste de uma vez.}
            \end{itemize}
    \end{itemize}