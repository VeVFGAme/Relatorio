% Definindo novas cores
\definecolor{verde}{rgb}{0,0.5,0}
% Configurando layout para mostrar codigos C++
\lstset{
  language=C++,
  basicstyle=\ttfamily\small,
  keywordstyle=\color{blue},
  stringstyle=\color{verde},
  commentstyle=\color{red},
  extendedchars=true,
  showspaces=false,
  showstringspaces=false,
  numbers=left,
  numberstyle=\tiny,
  breaklines=true,
  backgroundcolor=\color{green!10},
  breakautoindent=true,
  captionpos=b,
  xleftmargin=0pt,
}
\pagestyle{empty}

\chapter{Desenvolvimento}

Neste capítulo é apresentado a execução dos testes realizados na \textbf{FGAme}.

\section{Instalação da \textbf{FGAme}}
Para a instalação da \textbf{FGAme}, é necessário a utilização da linguagem \textit{Python} na versão 3.4 ou superior, além da utilização do programa de gerência de dependência \textit{PIP}.

Este procedimento funcionara em sistemas \textit{debian based}, tais comandos devem ser executados como administrador, então deve-se colocar \textit{sudo}, sempre que solicitado.

\begin{lstlisting}
$ apt-get update
$ apt-get -y install python3-pip
$ apt-get -y install python3-dev
$ apt-get -y install python3-pil
$ apt-get -y python3-numpy 
\end{lstlisting}

Após instalar as o \textit{python3} com o \textit{pip}, prossegue-se com a instalação do modulo \textbf{FGAme}, bem como o \textit{Framework} de testes unitários \textit{pygame}, que deve ser baixado e compilado localmente, utiliza-se o \textit{mercurial} ferramenta de controle de versão semelhante ao \textit{git}, clona-se o repositório da \textbf{FGAme}, e prossegue-se com a instalação.

\begin{lstlisting}
$ python3 -m pip install FGAme
$ apt-get -y install mercuria
$ hg clone https://bitbucket.org/pygame/pygame
$ python3 pygame/setup.py build
$ python3 pygame/setup.py install
\end{lstlisting}

Continuando instala-se o conjunto de dependências adicionais para o funcionamento da \textbf{FGAme}.

\begin{lstlisting}
$ apt-get update
$ apt-get install -y libsdl-image1.2-dev libsdl-mixer1.2-dev libsdl-ttf2.0-dev libsmpeg-dev libsdl1.2-dev libportmidi-dev libswscale-dev libavformat-dev libavcodec-dev libfreetype6-dev
$ hg clone https://bitbucket.org/pygame/pygame
$ pip3 install pgzero
\end{lstlisting}

Finalizando instala-se os módulos de requisitos para desenvolver na \textbf{FGAme}.

\begin{lstlisting}
$ apt-get update
$ pip install image
$ pip install -r requirements-dev.txt
$ pip install -r requirements.txt
\end{lstlisting}

Após a execução destes passos a \textit{engine} provavelmente já estará funcionando normalmente em sua totalidade.

\section{Execução dos testes}

Para a realização dos testes, foi inicialmente marcado uma reunião com o Prof.Fábio Macedo Mendes, responsável pelo desenvolvimento da \textbf{FGAme}, nesta reunião o professor concedeu um \textit{overview} sobre a \textbf{FGAme}, bem como a realização de testes na mesma, descrevendo as principais classes e módulos existentes no \textit{software}.

Nesta reunião o professor também expôs algumas classes que deveriam demandar uma atenção maior por parte da equipe de testes. 

Na execução dos testes houve uma série de imprevistos, que ocasionou em diversos problemas para a equipe de testadores, sendo eles:

\begin{itemize}
\item Código altamente complexo.
\item Pouco conhecimento e pouca informação sobre o funcionamento da \textit{FGAme}.
\item A ferramenta responsável por gerar a cobertura apresentou diversos erros, gerando algumas informações inconsistentes.
\item Problemas na configuração do ambiente de desenvolvimento por parte de alguns testadores.
\end{itemize}


Na realização dos testes focou-se em classes que apresentavam uma melhor legibilidade, além de evidentemente focou-se em classes que poderiam aumentar de maneira eficiente a cobertura do código.

No modulo responsável pelos cálculos envolvendo física, \textit{physics} executou-se os testes nas seguintes classes. 
\begin{itemize}
\item simulation.py
\item body.py
\end{itemize}

Além de realizar os testes nas seguintes classes.
\begin{itemize}
\item conf.py
\item objects.py
\item text.py
\item tracker.py
\end{itemize}
