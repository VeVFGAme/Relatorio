\chapter[Introdução]{Introdução}
\textbf{FGAme} é um motor de jogo focado em física, ou pelo termo em inglês , \textit{game} \textit{engine}, que é um \textit{framework} no qual visa simplificar o desenvolvimento de jogos eletrônicos.

O projeto foi desenvolvido pelo Professor Dr. Fábio Macêdo Mendes da Faculdade do Gama - \textbf{UnB} para ser utilizado na disciplina de Física para Jogos para o ensino de física aplicado a jogos. Porém como o principal desenvolvedor do motor não é da área de Engenharia de Software, isso foi um dos motivos para o projeto ser desenvolvido sem a devida atenção em Verificação e Validação.

Ao início do projeto a \textit{engine} apresentava em torno de 30\% de cobertura de testes unitários e com a \textit{build} da ferramenta de integração contínua Travis falhando. Também não foi feito nenhum tipo de análise para verificar a qualidade dos testes unitários implementados.



Esse software pode ser acessado no seguinte \href{https://github.com/fabiommendes/FGAme}{endereço}%
\footnote{https://github.com/fabiommendes/FGAme}

\section{Problema}

O maior problema encontrado no software, é o fato da cobertura de testes se encontra insatisfatória para garantir que a \textit{engine} seja realmente estável a ponto de ser natural a contribuição para a mesma como um projeto de software livre.
    
Além dos problemas relativos à falta de cobertura de teste, existe ainda a incerteza da qualidade dos testes já implementados, não garantindo que as funcionalidades as quais os testes apontam estejam realmente estáveis.

Como prova deste problema, podemos perceber que a integração contínua no software está quebrando devido a complexidade do projeto, não sendo possível ter garantia que o software continua funcionando quando o mesmo é evoluído.

Em resumo, os problemas encontrados no projeto com relação a qualidade são:
\begin{itemize}
\item Baixa Verificação dos Testes unitários.
\item Incerteza da qualidade dos testes já implementados.
\item Integração contínua no software está quebrando.
\item Cobertura de testes se encontra insatisfatória.
\end{itemize}


\section{Objetivos Gerais}
\begin{itemize}
    \item Melhorar a confiabilidade dos testes implementados.
    \item Verificar a integridade da \textit{build} de \textit{software}.
\end{itemize}

\section{Objetivos Específicos }
\begin{itemize}
    \item Aumentar a cobertura de testes.
    \item Otimizar a verificação os testes unitários no momento da integração.
\end{itemize}

\section{Questões de Pesquisa}
\begin{itemize}
    \item Como aumentar a confiabilidade dos testes unitários?
    \item Como garantir uma melhor qualidade das \textit{build} de software?
    \item Testes em \textit{python}?
\end{itemize}

\subsection{Metodologia}
\begin{itemize}
    \item Definição de palavras-chave
    \item Realizar pesquisa manuais nas principais bases de dados,\textit{ Scopus, IEEEXplorer, ACM.}
    \item \textit{Snowball} nos artigos com grande relevância.
\end{itemize}

\section{Resultados Esperados}
\subsection{Abordagem}
\begin{itemize}
    \item Garantir a verificação automática dos testes unitários, através da correção das falhas existentes na Integração continua, melhorando assim a entrega das \textit{Build}'s do Software.  
    \item Aumento da garantia de qualidade do software, com o incremento da cobertura de testes de 30\% para 55\%.
\end{itemize}

\subsection{Métodos}
\begin{itemize}
    \item Realizar testes unitários para aumentar a cobertura de software da \textbf{FGAme}.
\end{itemize}

\subsection{Técnicas}
\begin{itemize}
    \item Testes caixa branca.
\end{itemize}

\subsection{Ferramentas}
As ferramentas utilizadas neste trabalho.
\begin{itemize}
    \item \href{http://doc.pytest.org/en/latest/}{pytest}\footnote{http://doc.pytest.org/en/latest/}: Ferramenta utilizada para a realização dos testes unitários.
    \item \href{https://travis-ci.org/fabiommendes/FGAme}{Travis}\footnote{https://travis-ci.org/fabiommendes/FGAme}: Ferramenta utilizada a verificação da Build, através da Integração Continua.
\end{itemize}
