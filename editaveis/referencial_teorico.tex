\chapter{Referencial Teórico}

	Como técnica de levantamenteo do referencial teórico foi utilizada a técnica de revisão de literatura. Segundo \citeonline{revisao_de_literatura} uma revisão de literatura consiste em no processo de investigação, com o levantamento de um problema, tema, e a realização do levantamento bibliográfico  do assunto tratado. Assim em relação ao tema apresentado neste trabalho, a revisão literaria consistitiu no levantamento de palavras-chave, que tinham a finalidade de  sanar as lacunas com relação a falta de testes unitários, bem como a utilização de verificação e validação no desenvolvimento dos testes..

    logo após ser feita uma síntese dos textos estudados fica mais simples de se obter um entendimento prévio a respeito da área estudada  \apudonline{livro_sistematizacao_conhecimento}{revisao_de_literatura}.
    Para este trabalho o propósito da revisão será ganhar perspectivas metodológicas e desta forma contemplará os conceitos de:

    \begin{itemize}

        \item Verificação e Validação
        \item Testes Unitários;
        \item Cobertura de Testes;
        \item Escrita de Testes.
        \item Qualidade de Testes.
    \end{itemize}

\vfill
\pagebreak

\section{Teste de \textit{Software}}
   
\subsection{Verificação e Validação}

A verificação e validação é considerada amplamente uma das disciplinas mais importantes da engenharia de Software, tendo por exemplo a sua obrigatoriedade nos mais importantes modelos de maturidade encontrados no mercado com o \textit{CMMI} e o \textit{MPS-BR}, sendo assim conceitos fundamentais na desenvolvimento de \textit{Software}.

Elas quando seguidas, possibilitam que os diversos produtos gerados ao decorer do projeto cumprão aquilo a que foram destinado a cumprir, com enfoque prinsipalmente em  aspectos de qualidade otimização de produtos.
.\cite{gregor}

E como o retrabalho tende a crescer ao longo do desenvolvimento do projeto e como a \textit{FGAme} é relativamente nova e pretende ser mantida pelo professor Dr. Fábio Macedo, o aumento da cobertura de testes e o aprimoramento dos testes que já estavam implementados vem para facilitar a contribuição e a estabilidade da \textit{engine}.


\subsection{Teste de \textit{sofware}}
	De acordo com \cite{IEEE}, teste é descrito como a atividade na qual, execulta-se sobre condições especificas, com resultados planejados, produtos, componentes ou sistemas. Deste modo, nesta execução controlada tem-se o intuito de validar ou verificar os resultados.



Segundo \cite{artigo_intro_teste}, o teste de software é uma investigação conduzida para confirmar se o produto atingiu ou não as especificações no qual o produto foi definido. O teste de software também pode fornece uma visão objetiva e independente do software para o negócio, nisto tem-se uma melhor compreendensão dos riscos na implementação. 

Novamente \cite{artigo_intro_teste} descreve que, as técnicas de teste incluem, mas não estão limitados ao processo de execução de um programa, ou com a intenção de encontrar erros de software. Teste de software é qualquer atividade destinada a avaliar um atributo ou capacidade de um programa ou sistema e Determinando que ele atende aos resultados exigidos. Porém \cite{artigo_intro_teste} salienta que um bom teste é aquele que tem uma maior probalilidade de encontrar erros.

Embora crucial para a qualidade do software, o teste de software continua a ser uma arte, devido à limitada compreensão dos princípios de Programas. 
 O objetivo do teste pode ser a qualidade de garantia, verificação e validação, ou estimativa de confiabilidade. 

Teste de Software é um elemento crítico para a qualidade do produto, pois representa a revisão final da especificação, projeto e geração de código. Embora durante todo o processo de desenvolvimento de software sejam utilizadas técnicas, métodos e ferramentas a fim de evitar que erros sejam introduzidos no produto, a atividade de teste continua sendo de fundamental importância para a eliminação dos erros que persistem \cite{maldonado}.

    \subsection{Testes Unitários}
        O teste unitário tem como foco a menor unidade do projeto de software, procura erros de lógica e implementação nos módulos de forma separada. (R. S. Pressman) com adaptações.

         Teste de Unidade também conhecido como testes unitários. Tem por objetivo explorar a menor unidade do projeto, procurando provocar falhas ocasionadas por defeitos de lógica e de implementação em cada módulo, separadamente. O universo alvo desse tipo de teste são os métodos dos objetos ou mesmo pequenos trechos de código. (NETO;DIAS,2007)
        
        

    \subsection{Testes de Integração}

        Teste de Integração: visa provocar falhas associadas às interfaces entre os módulos quando esses são integrados para construir a estrutura do software que foi estabelecida na fase de projeto 
        (NETO;DIAS,2007)

        Manter um ambiente de testes e integração contínua, é imprescindível para qualquer projeto ou desenvolvedor sério, que queira manter sua produtividade em alta e foco na entrega de valor de cada demanda, e não em tarefas repetitivas e passíveis de erro.(TRAVIS CI, 2016)


    \subsection{\textit{Framework} de Testes}
        
