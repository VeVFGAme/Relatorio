\chapter{Referencial Teórico}

    Para realizar o levantamento do referencial teórico foi utilizada a técnica de revisão de literatura.
    Segundo \citeonline{revisao_de_literatura}, uma revisão de literatura comporta uma parte muito importante no processo de investigação, 
    uma vez estabelecida por localizar, analisar, sintetizar e interpretar a investigação prévia referente a uma área de estudo. 
    Além disso, uma boa revisão de literatura auxilia no entendimento de um problema e desenvolvimento dos conhecimentos, 
    conforme \apudonline{livro_sistematizacao_conhecimento}{revisao_de_literatura}, “cada investigador analisa minuciosamente os trabalhos dos investigadores 
    que o precederam e, só então, compreendido o testemunho que lhe foi confiado, parte equipado para a sua própria aventura”, 
    ou seja, ao se iniciar o processo de revisão de literatura raramente não existirão assuntos abordados no passado, sobre o 
    mesmo tema ou similares, que auxiliarão como fonte de conhecimento para a atual pesquisa.

    Novamente, segundo \citeonline{revisao_de_literatura}, os propósitos da revisão de literatura num estudo de investigação são:

    \begin{itemize}

        \item \textbf{Delimitar o problema de investigação:} formular uma definição concreta sobre o que será investigado visando não comprometer todo o trabalho com problemas mal delimitados;
        \item \textbf{Procurar novas linhas de investigação:} consiste em entender o que já foi realizado sobre um determinado problema e buscar pelas suas áreas que ainda não foram ou foram pouco exploradas;
        \item \textbf{Evitar abordagens infrutíferas:} buscar que as linhas de investigação definidas sejam proveitosas e possuam resultados significativos;
        \item \textbf{Ganhar perspectivas metodológicas:} não rever apenas os resultados do estudo, mas sim, realizar uma leitura geral sobre todos os tópicos abordados;
        \item \textbf{Identificar recomendações para investigações futuras:} após finalizado o estudo, identificar novas questões e sugestões para investigações futuras.

    \end{itemize}

\vfill
\pagebreak

    Para este trabalho o propósito da revisão será ganhar perspectivas metodológicas e desta forma contemplará os conceitos de:

    \begin{itemize}

        \item Verificação e Validação
        \item Testes Unitários;
        \item Testes de Integração;
        \item \textit{Framework} de Testes.

    \end{itemize}


\section{Teste de \textit{Software}}
   
\subsection{Verificação e Validação}

A verificação e validação são chave no desenvolvimento de software.
Elas garantem que o produto cumpra o que esta destinado a cumprir,
preservando sobretudo aspectos de qualidade e respectivamente sua melhoria contínua.
Desta forma, garantem a detecção prematura de falhas no processo de desenvolvimento,
bem como reduzem o tempo necessário para remover essas falhas e ajudam a produzir um
produto mais robusto.


E como o retrabalho tende a crescer ao longo do desenvolvimento do projeto e como a engine é relativamente nova e pretende ser mantida pelo professor Dr. Fábio Macedo, o aumento da cobertura de testes e o aprimoramento dos testes que já estavam implementados vem para facilitar a contribuição e a estabilidade da engine.

\subsection{Teste de \textit{sofware}}
O teste de software é uma investigação conduzida levar os interessados de informações sobre a qualidade do o produto ou serviço sob teste. 
O teste de software também fornece uma visão objetiva e independente do software para
O negócio para apreciar e compreender os riscos na implementação do software. 
As técnicas de teste incluem, mas não estão limitados ao processo de execução de um programa de aplicação, ou com a intenção de encontrar erros de software.
 Teste de software é qualquer atividade destinada a avaliar um atributo ou capacidade de um programa ou sistema e Determinando que ele atende aos resultados exigidos. 
Embora crucial para a qualidade do software e amplamente Programadores e testadores, o teste de software continua a ser uma arte, devido à limitada compreensão dos princípios de Programas. 
 O objetivo do teste pode ser a qualidade de garantia, verificação e validação, ou estimativa de confiabilidade. 
Priyanka e Vipin Arora  com adaptações (tradução nossa).

Teste de Software é um elemento crítico para a qualidade do produto, pois representa a revisão final da especificação, projeto e geração de código. Embora durante todo o processo de desenvolvimento de software sejam utilizadas técnicas, métodos e ferramentas a fim de evitar que erros sejam introduzidos no produto, a atividade de teste continua sendo de fundamental importância para a eliminação dos erros que persistem (Maldonado, 1991).

    \subsection{Testes Unitários}
        O teste unitário tem como foco a menor unidade do projeto de software, procura erros de lógica e implementação nos módulos de forma separada. (R. S. Pressman) com adaptações.

         Teste de Unidade também conhecido como testes unitários. Tem por objetivo explorar a menor unidade do projeto, procurando provocar falhas ocasionadas por defeitos de lógica e de implementação em cada módulo, separadamente. O universo alvo desse tipo de teste são os métodos dos objetos ou mesmo pequenos trechos de código. (NETO;DIAS,2007)
        
        

    \subsection{Testes de Integração}
        

    \subsection{\textit{Framework} de Testes}
        

        \begin{itemize}

        \item Ajudar na melhoria de qualidade;
        \item Ajudar no entendimento de um Sistema Em Teste (SUT);
        \item Reduzir riscos;
        \item Ser fáceis de serem escritos, executados e mantidos;
        \item Possuir manutenção mínima conforme a evolução de seu sistema.

        \end{itemize}

        Para \citeonline{livro_x_unit}, escrever e executar testes automatizados envolve várias etapas que geravam um processo “tedioso, consumidor de tempo, propenso a erros e caro”, além disso, muitas dessas etapas eram repetidas dentre os diversos testes, com isso em mente, uma maneira eficiente de minimizar todo este esforço gerado consistia na utilização de um \textit{framework} de automação de testes. Segundo o autor, esse \textit{framework} deveria realizar a suíte de testes e o registro de seus resultados pelo meio de alguns mecanismos, os quais consistiam em:

        \begin{itemize}

        \item Habilidade de encontrar testes individuais;
        \item Reunir os testes individuais em uma suíte de testes;
        \item Executar cada teste individualmente;
        \item Verificar os resultados esperados;
        \item Coletar e reportar qualquer falha ou erro;
        \item Realizar uma limpeza quando ocorrerem falhas ou erros.

        \end{itemize}

        Todas essas necessidades são retratadas bem por \citeonline{livro_x_unit}, o qual salienta que a utilização de um \textit{framework} de testes ajuda a isolar a lógica necessária para realizar suas execuções de suas lógicas funcionais, além disso, tal abordagem pode ajudar na redução de duplicação de código e minimização na ocorrência de “testes obscuros”, os quais são difíceis de serem entendidos e manutenidos, podendo deixar com que bugs passem despercebidos.
