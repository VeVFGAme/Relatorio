\chapter{Conclusão}

Com base neste estudo verificou-se a enorme complexidade da disciplina de Verificação e Validação, bem como a sua gigantesca importância não somente para o software em si, mas para todo o processo de desenvolvimento do mesmo, isto foi constatado ao realizar os testes na \textbf{FGAme}, a qual, a pesar da alta complexidade, não possuía nenhuma disciplina em relação a Verificação e Validação, isto acabou gerando um esforço extremamente grande por parte da equipe de testadores, porém pouco resultado efetivo foi gerado no processo de teste da \textbf{FGAme}.

Houve também um grande aprendizado na equipe em relação as definições de metas e objetivos a serem seguidos em um projeto de Verificação e Validação, isso ficou evidente na definição de um número mágico em relação ao objetivo a ser alcançado na cobertura de testes, ficou notório que a definição deste número ocorreu sem nenhuma base acadêmica em relação ao mesmo, com somente a experiência dos graduandos, aumenta a probabilidade de acontecer um erro de planejamento.

Além disto, muitos cenários adversos foram encontrados ao longo da realização do projeto, como a alta complexidade das funções da \textit{engine}, isto acabou gerando um grande gargalo entre os testadores, que necessitavam primeiramente gastar uma grande quantia de tempo para entender as classes e métodos do projeto, uma vez que a \textbf{FGAme} trata de eventos físicos, que não são fáceis de entender, como colisão, espaço vetorial, planos geométricos, integrais, vetores multidimensionais, simulações de objetos reais, entre outros.

Porém mesmo com todos os cenários adversos encontrados, a equipe conseguiu atingir seus objetivos propostos neste trabalho, com a porcentagens de cobertura em 57\%, uma grande melhora em relação ao estado inicial encontrado na \textbf{FGAme}. Além da automatização da \textit{Build}, a qual foi configurada corretamente, corrigindo os erros existentes. 

Ficando como proposta para trabalhos futuros, uma pesquisa detalhada sobre todos os módulos da \textbf{FGAme}, para que assim possa possibilitar a realização de testes outros tipos de testes complementares como Integração e Desempenho.
