% Definindo novas cores
\definecolor{verde}{rgb}{0,0.5,0}
% Configurando layout para mostrar codigos C++
\lstset{
  language=C++,
  basicstyle=\ttfamily\small,
  keywordstyle=\color{blue},
  stringstyle=\color{verde},
  commentstyle=\color{red},
  extendedchars=true,
  showspaces=false,
  showstringspaces=false,
  numbers=left,
  numberstyle=\tiny,
  breaklines=true,
  backgroundcolor=\color{green!10},
  breakautoindent=true,
  captionpos=b,
  xleftmargin=0pt,
}
\pagestyle{empty}

\chapter{Desenvolvimento}

Neste capítulo é apresentado a execusão dos testes realizados na FGAme.

\section{Instalação da FGAme}
Para a instalação da FGAme, é nescessário a utilização da linguagem \textit{Python} na versão 3.4 ou superior, além da utilização do programa de gerência de dependencia \textit{PIP}.

Este procedimento funcionara em sistemas \textit{debian based}, tais comandos devem ser execultados como administrador, então deve-se colocar \textit{sudo}, sempre que solicitado.

\begin{lstlisting}
$ apt-get update
$ apt-get -y install python3-pip
$ apt-get -y install python3-dev
$ apt-get -y install python3-pil
$ apt-get -y python3-numpy 
\end{lstlisting}

Após instalar as o python3 com o pip, prosegue-se com a instalação do modulo FGAme, bem como o \textit{Framework} de testes unitários \textit{pygame}, que deve ser baixado e compilado localmente, utiliza-se o \textit{mercurial} ferramenta de controle de versão semelhante ao git, clona-se o repositorio da FGAme, e prosegue-se com a instalação.

\begin{lstlisting}
$ python3 -m pip install FGAme
$ apt-get -y install mercuria
$ hg clone https://bitbucket.org/pygame/pygame
$ python3 pygame/setup.py build
$ python3 pygame/setup.py install
\end{lstlisting}

Continuando instala-se o conjunto de dependencias adicionais para o funcionamento da FGAme.

\begin{lstlisting}
$ apt-get update
$ apt-get install -y libsdl-image1.2-dev libsdl-mixer1.2-dev libsdl-ttf2.0-dev libsmpeg-dev libsdl1.2-dev libportmidi-dev libswscale-dev libavformat-dev libavcodec-dev libfreetype6-dev
$ hg clone https://bitbucket.org/pygame/pygame
$ pip3 install pgzero
\end{lstlisting}

Finalizando instala-se os modulos de requisitos para desenvolver na FGAme.

\begin{lstlisting}
$ apt-get update
$ pip install -r requirements-dev.txt
$ pip install -r requirements.txt
\end{lstlisting}

Após a execução destes passos a engine já provavelmente estará funcionando normalmente em sua totalidade.

\section{Execução dos testes}

