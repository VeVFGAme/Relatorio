\chapter{Referencial Teórico}

    Para realizar o referencial teórico foi utilizada a técnica de revisão de literatura.
    a revisão de literatura consiste em no processo de investigação, a respeito do tema apresentado aqui neste trabalho.

    logo após ser feita uma sintese dos textos estudados fica mais simples de se obter um entendimento previo a respeito da área estudada.
    Para este trabalho o propósito da revisão será ganhar perspectivas metodológicas e desta forma contemplará os conceitos de:

    \begin{itemize}

        \item Verificação e Validação
        \item Testes Unitários;
        \item Testes de Integração;
        \item \textit{Framework} de Testes.

    \end{itemize}

\vfill
\pagebreak

\section{Teste de \textit{Software}}
   
\subsection{Verificação e Validação}

A verificação e validação são chave no desenvolvimento de software.
Elas garantem que o produto cumpra o que esta destinado a cumprir,
preservando sobretudo aspectos de qualidade e respectivamente sua melhoria contínua.
Desta forma, garantem a detecção prematura de falhas no processo de desenvolvimento,
bem como reduzem o tempo necessário para remover essas falhas e ajudam a produzir um
produto mais robusto.


E como o retrabalho tende a crescer ao longo do desenvolvimento do projeto e como a engine é relativamente nova e pretende ser mantida pelo professor Dr. Fábio Macedo, o aumento da cobertura de testes e o aprimoramento dos testes que já estavam implementados vem para facilitar a contribuição e a estabilidade da engine.

\subsection{Teste de \textit{sofware}}
O teste de software é uma investigação conduzida levar os interessados de informações sobre a qualidade do o produto ou serviço sob teste. 
O teste de software também fornece uma visão objetiva e independente do software para
O negócio para apreciar e compreender os riscos na implementação do software. 
As técnicas de teste incluem, mas não estão limitados ao processo de execução de um programa de aplicação, ou com a intenção de encontrar erros de software.
 Teste de software é qualquer atividade destinada a avaliar um atributo ou capacidade de um programa ou sistema e Determinando que ele atende aos resultados exigidos. 
Embora crucial para a qualidade do software e amplamente Programadores e testadores, o teste de software continua a ser uma arte, devido à limitada compreensão dos princípios de Programas. 
 O objetivo do teste pode ser a qualidade de garantia, verificação e validação, ou estimativa de confiabilidade. 
Priyanka e Vipin Arora  com adaptações (tradução nossa).

Teste de Software é um elemento crítico para a qualidade do produto, pois representa a revisão final da especificação, projeto e geração de código. Embora durante todo o processo de desenvolvimento de software sejam utilizadas técnicas, métodos e ferramentas a fim de evitar que erros sejam introduzidos no produto, a atividade de teste continua sendo de fundamental importância para a eliminação dos erros que persistem (Maldonado, 1991).

    \subsection{Testes Unitários}
        O teste unitário tem como foco a menor unidade do projeto de software, procura erros de lógica e implementação nos módulos de forma separada. (R. S. Pressman) com adaptações.

         Teste de Unidade também conhecido como testes unitários. Tem por objetivo explorar a menor unidade do projeto, procurando provocar falhas ocasionadas por defeitos de lógica e de implementação em cada módulo, separadamente. O universo alvo desse tipo de teste são os métodos dos objetos ou mesmo pequenos trechos de código. (NETO;DIAS,2007)
        
        

    \subsection{Testes de Integração}

        Teste de Integração: visa provocar falhas associadas às interfaces entre os módulos quando esses são integrados para construir a estrutura do software que foi estabelecida na fase de projeto 
        (NETO;DIAS,2007)

        Manter um ambiente de testes e integração contínua, é imprescindível para qualquer projeto ou desenvolvedor sério, que queira manter sua produtividade em alta e foco na entrega de valor de cada demanda, e não em tarefas repetitivas e passíveis de erro.(TRAVIS CI, 2016)


    \subsection{\textit{Framework} de Testes}
        
